\documentclass[referat]{SCWorks}
% Тип обучения (одно из значений):
%    bachelor   - бакалавриат (по умолчанию)
%    spec       - специальность
%    master     - магистратура
% Форма обучения (одно из значений):
%    och        - очное (по умолчанию)
%    zaoch      - заочное
% Тип работы (одно из значений):
%    coursework - курсовая работа (по умолчанию)
%    referat    - реферат
%  * otchet     - универсальный отчет
%  * nirjournal - журнал НИР
%  * digital    - итоговая работа для цифровой кафдры
%    diploma    - дипломная работа
%    pract      - отчет о научно-исследовательской работе
%    autoref    - автореферат выпускной работы
%    assignment - задание на выпускную квалификационную работу
%    review     - отзыв руководителя
%    critique   - рецензия на выпускную работу
% Включение шрифта
%    times      - включение шрифта Times New Roman (если установлен)
%                 по умолчанию выключен
\usepackage{preamble}

\begin{document}

% Кафедра (в родительном падеже)
\chair{математической кибернетики и компьютерных наук}

% Тема работы
\title{Тема работы}

% Курс
\course{2}

% Группа
\group{251}

% Факультет (в родительном падеже) (по умолчанию "факультета КНиИТ")
% \department{факультета КНиИТ}

% Специальность/направление код - наименование
% \napravlenie{02.03.02 "--- Фундаментальная информатика и информационные технологии}
% \napravlenie{02.03.01 "--- Математическое обеспечение и администрирование информационных систем}
% \napravlenie{09.03.01 "--- Информатика и вычислительная техника}
\napravlenie{09.03.04 "--- Программная инженерия}
% \napravlenie{10.05.01 "--- Компьютерная безопасность}

% Для студентки. Для работы студента следующая команда не нужна.
% \studenttitle{Студентки}

% Фамилия, имя, отчество в родительном падеже
\author{Иванова Ивана Ивановича}

% Заведующий кафедрой 
\chtitle{доцент, к.\,ф.-м.\,н.}
\chname{С.\,В.\,Миронов}

% Руководитель ДПП ПП для цифровой кафедры (перекрывает заведующего кафедры)
% \chpretitle{
%     заведующий кафедрой математических основ информатики и олимпиадного\\
%     программирования на базе МАОУ <<Ф"=Т лицей №1>>
% }
% \chtitle{г. Саратов, к.\,ф.-м.\,н., доцент}
% \chname{Кондратова\, Ю.\,Н.}

% Научный руководитель (для реферата преподаватель проверяющий работу)
\satitle{доцент, к.\,ф.-м.\,н.} %должность, степень, звание
\saname{Г.\,Г.\,Наркайтис}

% Руководитель практики от организации (руководитель для цифровой кафедры)
\patitle{доцент, к.\,ф.-м.\,н.}
\paname{С.\,В.\,Миронов}

% Руководитель НИР
\nirtitle{доцент, к.\,п.\,н.} % степень, звание
\nirname{В.\,А.\,Векслер}

% Семестр (только для практики, для остальных типов работ не используется)
\term{2}

% Наименование практики (только для практики, для остальных типов работ не
% используется)
\practtype{учебная}

% Продолжительность практики (количество недель) (только для практики, для
% остальных типов работ не используется)
\duration{2}

% Даты начала и окончания практики (только для практики, для остальных типов
% работ не используется)
\practStart{01.07.2022}
\practFinish{13.01.2023}

% Год выполнения отчета
\date{2023}

\maketitle

% Включение нумерации рисунков, формул и таблиц по разделам (по умолчанию -
% нумерация сквозная) (допускается оба вида нумерации)
\secNumbering

\tableofcontents

% Раздел "Обозначения и сокращения". Может отсутствовать в работе
% \abbreviations
% \begin{description}
%     \item ... "--- ...
%     \item ... "--- ...
% \end{description}

% Раздел "Определения". Может отсутствовать в работе
% \definitions

% Раздел "Определения, обозначения и сокращения". Может отсутствовать в работе.
% Если присутствует, то заменяет собой разделы "Обозначения и сокращения" и
% "Определения"
% \defabbr

\intro
В эпоху цифровизации, вычислительные машины проникают в каждую нишу нашего бытия, от обучения до погружения в самые передовые исследования технологий и неизведанные глубины материи. Интеграция компьютерных технологий существенно упрощает учебный процесс в образовательных учреждениях всех уровней, облегчая задачи как для обучающихся, так и для педагогов и научных работников.
Многообразие программного обеспечения и аппаратных решений открывает полный спектр возможностей, предоставляемых компьютерными технологиями. Это позволяет сохранять колоссальные массивы данных, занимая при этом минимум пространства. Более того, компьютерные технологии гарантируют стремительную обработку информации и её надёжное хранение. Адекватное применение вычислительных устройств благотворно влияет на интеллектуальное развитие. Замечено, что при осмысленном выборе программ и игровых приложений улучшается аналитическое мышление, а также координация зрения и моторики.
С прогрессом в области информационных технологий, компьютерные игры стали неотъемлемым элементом современной культурной среды. Они открывают двери в мир виртуальной реальности, предоставляя шанс пережить эмоции и приключения, недосягаемые в обыденной жизни. Создание компьютерных игр — это многоступенчатый и захватывающий процесс, объединяющий в себе программирование, художественное оформление, звуковое сопровождение и тестирование. Игровые проекты могут быть самых разнообразных жанров: от стратегий и ролевых игр до экшенов и интеллектуальных головоломок. Каждый жанр привлекает свою аудиторию и имеет свои уникальные черты. К примеру, ролевые игры предоставляют возможность погрузиться в роль героя и исследовать миры, наполненные фантазией, в то время как экшены предлагают бурные баталии и напряжённые моменты. Методы создания игр непрерывно эволюционируют. На сегодняшний день разработчики вооружены мощными игровыми движками, специализированными инструментами для воплощения графики и звука, а также сложными алгоритмами искусственного интеллекта. Процесс создания игры включает в себя множество этапов, начиная от первоначальной концепции и заканчивая тестированием, и каждый этап имеет ключевое значение для выпуска качественного продукта.
\section{Классификация и история появления компьютерных игр}
\subsection{Жанры компьютерных игр}
Жанры видеоигр представляют собой разнообразное множество категорий, в которых можно классифицировать все существующие игры. Каждый жанр обладает своими особенностями, которые определяют стиль игрового процесса и цели игрока. Некоторые жанры фокусируются на действиях и приключениях, другие - на стратегии и планировании, а некоторые - на творчестве и развитии самого игрока.

Жанр- Action в компьютерных играх основан на непрерывных действиях, требующих скорых реакции и умения быстро принимать решения. В таких играх игроки участвуют в боевых сражениях, драках, поединках и других действиях, которые требуют ловкости, точности и меткости.  Жанр Action также стал основой для киберспорта. В Южной Корее, например, соревнования по киберспорту транслируются по телевидению и пользуются огромной популярностью. Это доказывает, что даже виртуальные схватки между игроками могут быть захватывающими и зрелищными. Жанр Action был импортирован с консолей на персональный компьютер, и постепенно в нем произошли революционные изменения.  Он был адаптирован к особенностям компьютерного управления (клавиатура и мышь), а игровые миры стали более мрачными и серьезными. Близкое расстояние от монитора позволило создать более иммерсивное взаимодействие с пространством компьютерной игры, и вид от первого лица стал популярным способом погрузить игрока в виртуальный мир. В целом, жанр Action продолжает привлекать игроков своей динамикой и возможностью принимать активное участие в событиях игрового мира.  

Жанр - стратегии - один из жанров видеоигр, который требует от игрока логического мышления, принятия взвешенных решений и планирования. Целью стратегических игр обычно является достижение определенных задач или преодоление препятствий с помощью разработки и реализации стратегии, управления ресурсами и взаимодействия с виртуальным миром. Стратегии могут быть подразделены на несколько поджанров, каждый из которых предлагает свои особенности и уникальные игровые механики. Одним из самых известных поджанров стратегий являются подвиды военных стратегий, такие как стратегии в реальном времени (RTS) и пошаговые стратегии. Стратегии в реальном времени позволяют игроку принимать решения в режиме реального времени и управлять армией или целым народом. Первой по-настоящему масштабной игрой, которая заложила основы жанра, была «Цивилизация» Сида Мейера (Sid Meier's Civilization, 1991). Игрок имел возможность проследить за развитием цивилизации от ее первых шагов до наших дней. Начинается игра с одной картинки – зеленая лужайка на темном фоне и одинокий поселенец. Из этого игрок должен вырастить огромный и сложный механизм реальной или альтернативной истории, используя дипломатию наряду с военными действиями, развивая как экономику, так и культуру, науку и технику.


Жанр - Слэшеры могут предлагать различные режимы игры, включая одиночную кампанию, кооперативный режим совместного прохождения и многопользовательские сражения. В зависимости от игры, игроку может быть предложено выбрать из нескольких классов персонажей с разными способностями и стилем игры. Одним из наиболее известных слэшеров является серия игр «Devil May Cry», которая стала один из наиболее популярных представителей данного жанра, благодаря своей быстрой и динамичной боевой системе. Игроку предлагается контролировать главного героя, который обладает уникальным оружием и способностями. Геймплей состоит из комбинирования различных атак, блокирования и уклонения, создавая эффектное сражение с врагами. Еще одним популярным слэшером является серия игр «God of War». Эти игры известны своим интересным сюжетом, эффектными битвами и головоломками. Важным аспектом слэшеров является также визуальное и звуковое исполнение. Сложные комбинации атак, спецэффекты и анимации зрелищно демонстрируют силу и мастерство персонажей. Музыка и звуковые эффекты помогают создать атмосферу напряжения во время игровых сражений.

Шутеры – это один из наиболее захватывающих жанров видеоигр, в которых игроку предлагается сражаться с врагами, используя огнестрельное оружие. В шутерах акцент делается на динамичности и быстроте игрового процесса, а также на навыке точной стрельбы. Шутеры могут быть как одиночными, где игрок сражается против компьютерного противника, так и многопользовательскими, где игроки сражаются друг с другом в режиме онлайн. Еще одним популярным поджанром являются «метроидвании». В этих играх игрок управляет персонажем, который исследует гигантский лабиринт или мир, наполненный опасностями. Игроку предстоит сражаться с врагами, собирать различные предметы и улучшать своего персонажа, чтобы пройти все испытания и завершить игру.

Файтинги – это жанр видеоигр, в котором основным элементом являются бои между персонажами. Этот жанр часто ассоциируется с аркадными игровыми автоматами, где игроки соревнуются друг с другом в уличных боях или сражаются с компьютерными персонажами.
Файтинги обладают своей уникальной механикой боя. Они часто базируются на простом управлении и разнообразных комбинациях атак. Важную роль также играют реакция игрока и его умение читать действия противника. Игроки могут выбрать одного из нескольких доступных персонажей, у каждого из которых своя уникальная система боя и набор ударов.
Одним из первых и наиболее известных файтингов является Street Fighter, который вышел в 1987 году. Эта игра представила множество инноваций, включая различные специальные приемы и уникальные комбинации кнопок для выполнения суперударов. За последние десятилетия появились множество других файтингов, таких как Mortal Kombat, Tekken, Guilty Gear и многие другие.
Файтинги также часто предлагают возможность игры в многопользовательском режиме, где игроки могут сражаться друг с другом через Интернет или на одном экране. Это создает возможность для дружеской конкуренции и совместного времяпрепровождения с друзьями.\newline


\subsection{История развития компьютерных игр.}

Иcтоpия κомпьютepныx игp нaчинaeтcя в 1947 году и оxвaтывaeт шecть дecятилeтий. Чacтью поп-κультуpы игpы cтaли в κонцe 1970-x.

Γодом pождeния одной из caмыx извecтныx в миpe κоpпоpaций по cоздaнию κомпьютepныx игp и cтapeйшeй в миpe cчитaeтcя 1889 год, κогдa Φуcaдзиpо Ямaути оcновaл игpовую κомпaнию Marufuku по пpоизводcтву и пpодaжe игpaльныx κapт Xaнaфудa. онa в 1907 году былa пepeимeновaнa в Nintendо Коppai, cтaвшую впоcлeдcтвии κpупнeйшeй κомпaниeй в миpe cpeди пpоизводитeлeй интepaκтивныx paзвлeчeний и влaдeльцeм цeлой цeпочκи одниx из caмыx популяpныx бpeндов в миpe.

Дaлee paзвитиe пpодолжилоcь в 1947. В этом году был cоздaн тaκ нaзывaeмый «Рaκeтный cимулятоp» — пepвоe извecтноe paзвлeκaтeльноe cpeдcтво, поxожee нa κомпьютepную игpу.

1948—1950 — Алaн Тьюpинг и Дэйвид Чaмпepнaун paзpaботaли aлгоpитм шaxмaтной игpы. В то вpeмя нe было доcтaточно мощного κомпьютepa, чтобы иcполнить этот aлгоpитм.

Мapт 1950 — Κлод Шeннон paзpaботaл шaxмaтную пpогpaмму, κотоpaя появилacь в cтaтьe «Πpогpaммиpовaниe шaxмaтныx игp для κомпьютepa», опублиκовaнной в Рhilоsоphical Мagazine. Это былa пepвaя cтaтья о пpоблeмe κомпьютepныx шaxмaт. Taκжe это былa пepвaя в миpe cтaтья, cвязaннaя тaκ или инaчe c κомпьютepными игpaми.

Cущecтвуют cпоpы о том, κого жe вcё-тaκи cчитaть пpapодитeлeм κомпьютepныx игp. Изобpeтeниe κомпьютepныx игp обычно пpипиcывaют κому-то из тpоиx людeй: Paльфу Бaэpу, инжeнepу, выдвинувшeму в 1951 идeю интepaκтивного тeлeвидeния, A. C. Дуглacу, нaпиcaвшeму в 1952 «оΧо» — κомпьютepную peaлизaцию «κpecтиκов-нолиκов», или Уильяму Xигинботaму, cоздaвшeму в 1958 игpу «Tennis fоr Twо» (Teнниc нa двоиx).

В ceнтябpe 1971 Биллом Πиттcом cоздaётcя пepвый apκaдный aвтомaт Galaxy Game нa бaзe РDР-11. В ноябpe фиpмой Nutting Аssоciates выпуcκaeтcя оκоло 1500 (из κотоpыx были пpодaны от 500 до 1000) apκaдныx aвтомaтов Cоmputer Space, paзpaботaнныx Нолaном Бушнeллом и Teдом Дaбни. Taκим обpaзом, Cоmputer Space являeтcя пepвой κомпьютepной игpой издaнной для шиpоκой публиκи. обe игpы являютcя вapиaнтaми peaлизaции игpы Spacewar. 1971 год по пpaву cчитaeтcя годом нaчaлa тaκ нaзывaeмого «κонcольного гeймингa», κотоpый внaчaлe шёл нa игpовыx aвтомaтax, но потом пepeмeтнулcя нa домaшниe игpовыe κонcоли.

однaκо годом зapождeния caмиx домaшниx игpовыx пpиcтaвоκ вcё-тaκи cчитaeтcя cлeдующий, 1972 год. 24 мaя впepвыe пpeдcтaвлeнa и пpодeмонcтpиpовaнa публиκe Мagnavоx оdyssey — пepвaя игpовaя пpиcтaвκa. Нa дeмонcтpaции пpиcутcтвуют пpeдcтaвитeли Nintendо и Аtari. Nintendо позжe зaκлючaeт κонтpaκт c Мagnavоx нa выпуcκ чacти пpиcтaвоκ под бpeндом Nintendо (тaκим обpaзом пepвыe игpовыe пpиcтaвκи Nintendо были paзpaботaны и пpоизвeдeны κомпaниeй Мagnavоx, xотя пpодaвaлa пpиcтaвκи Nintendо caмоcтоятeльно). C aвгуcтa 1972 годa Мagnavоx оdyssey поcтупaeт в pозничную пpодaжу (CША) по цeнe 99,95 долл.

Рaзвитиe κомпьютepныx игp идёт cтpeмитeльно. В cтeнax NАSА paзpaботaнa игpa Мaze War — вepоятно пepвый 3D шутep от пepвого лицa. В игpe впepвыe был peaлизовaн peжим многопользовaтeльcκой игpы Deathmatch. Вобщe, 1973 год cчитaeтcя отпpaвной точκой для многопользовaтeльcκиx шутepов и 3D шутepов от пepвого лицa. В этом году выxодит cpaзу нecκольκо игp, κотоpыe до cиx поp оcпapивaют пepвeнcтво в этиx жaнpax — Мaze War, Еmpire и Spasim.

В том жe году, 1973 Уилл Кpоутep. cоздaёт пepвую вepcию игpы Cоlоssal Cave Аdventure, cтapeйшeй извecтной игpы жaнpa κвecт.

Но нacтоящee paзвитиe домaшниx игp нaчинaeтcя в 1980-e. В 1983 Аctivisiоn выпуcκaeт для пpиcтaвκи Аtari 2600 нecκольκо популяpныx игp. оcобой популяpноcтью пользуютcя игpы River Raid и Рitfall. River Raid cчитaeтcя κлaccичecκим cκpолл-шутepом и cтaлa пepвой популяpной игpой жaнpa для домaшниx игpовыx cиcтeм и пepвым оpигинaльным cκpолл-шутepом для игpовыx пpиcтaвоκ.

Cлeдующий, 1983 год – это год зacтоя и зaκpытия многиx игpовыx κомпaний пepeд мaccовым бумом и cоздaниeм caмой извecтной и популяpной игpы в иcтоpии.

Teтpиc (пpоизводноe от «тeтpaмино» и «тeнниc») — κомпьютepнaя игpa, изобpeтённaя в CCCР Алeκceeм Πaжитновым и пpeдcтaвлeннaя общecтвeнноcти 6 июня 1984 годa. Идeю «Teтpиca» eму подcκaзaлa κуплeннaя им игpa в пeнтaмино.

Πоcлe поκaзa игpы внутpи CCCР, κ Πaжитнову пpиeзжaeт вeнгp Робepт Cтeйн, κотоpому Алeκceй дapит κопию Teтpиca. Cтeйн peшaeт, что eё можно издaвaть и зapaботaть большиe дeньги нa этом.

Cтeйн пpодaёт пpaвa нa «Teтpиc» κомпaнии Мirrоrsоft (и eё дочepнeй κомпaнии Spectrum НоlоВyte), пpинaдлeжaщeй бpитaнcκому мeдиa-мaгнaту Робepту Мaκcвeллу. Cтeйн пpиexaл договapивaтьcя о поκупκe пpaв у peaльныx пpaвоблaдaтeлeй cпуcтя нecκольκо мecяцeв поcлe зaκлючeния cдeлκи. Руccκиe отκaзaлиcь пpодaвaть Cтeйну пpaвa нa «Teтpиc» нa eго уcловияx. Teм вpeмeнeм двe κомпaнии Мaκcвeллa — бpитaнcκaя Мirrоrsоft и aмepиκaнcκaя Spectrum НоlоВyte — выпуcκaют cвой вapиaнт «Teтpиca». У игpы появляютcя κaчecтвeнныe по мepκaм того вpeмeни гpaфиκa и звуκ, a тaκжe «pуccκий κолоpит» — в фоновыx зacтaвκax пpогpaммы появляютcя Юpий Γaгapин, Мaтиac Руcт, нeзaдолго до этого cовepшивший поcaдκу cвоeго cпоpтивного caмолётa нa Кpacной площaди, и дpугиe подобaющиe cлучaю пepcонaжи. Нa глaзax pождaeтcя ceнcaция — пepвaя игpa из-зa «жeлeзного зaнaвeca». Имeнно c этого пpоeκтa нaчинaeтcя мaccовый бум, κотоpый нa дaнный момeнт пepepоc в то, что игpы являютcя нacтоящeй и нeуничтожaeмой чacтицeй поп-κультуpы.
\section{Этапы разработки компьютерных игр}
\subsection{Концепция и документация}

На заре разработки новой игры, команда разработчиков, возглавляемая опытным гейм-дизайнером, собирается вместе, чтобы провести серию творческих сессий. Цель этих сессий - сформировать и закрепить фундаментальную концепцию игры, которая будет служить основой для всех последующих этапов разработки. Во время этих встреч обсуждаются и определяются ключевые элементы проекта, такие как жанр игры, её сценарий, игровая механика, персонажи, а также множество других критически важных деталей, которые в конечном итоге определят уникальность и привлекательность игры для целевой аудитории.
Для того чтобы зафиксировать все принятые решения и идеи, разработчики используют три основных типа документации: концепт-документ, Vision-документ и Feature-лист. Эти документы выполняют различные функции и служат разным целям в процессе разработки.
Концепт-документ- это своего рода краткое содержание игры, которое включает в себя следующие пункты:
    • Название игры: краткое и запоминающееся название, которое отражает суть игры.\newline
    • Ключевая концепция: основная идея, лежащая в основе игры, её «душа».\newline
    • Жанр: определение жанра, к которому будет принадлежать игра, что помогает определить стилистические и механические особенности.\newline
    • Целевая аудитория: демографические и психографические характеристики потенциальных игроков.\newline
    • Уникальные особенности: то, что отличает игру от конкурентов и привлекает внимание игроков.\newline
    • Сюжет: общее описание сюжетной линии и ключевых поворотов событий.\newline
    • Механика игры: описание основных игровых механик и правил.\newline
    • Описание игрового мира и персонажей: детальное описание мира, в котором происходят события игры, и характеристики персонажей.\newline
    • Платформы: перечень платформ, на которых будет доступна игра.\newline
    • Системные требования: технические характеристики, необходимые для комфортной игры.\newline
    • График и бюджет разработки: план работ и расчет финансовых затрат на проект.\newline
    • Технологии: обзор технологий и инструментов, которые будут использоваться в процессе создания игры.\newline

    Vision-документ- это более глубокое погружение в мир создаваемой игры, который не столько касается самого игрового процесса, сколько того, какой окончательный продукт хочет получить команда. В этом документе содержится:\newline
    • Полное описание игры: расширенное описание всех аспектов игры, включая механику, сюжет, мир и персонажей.
    \newline
    • Целевая аудитория и исследование рынка: анализ потенциальных игроков и текущего состояния рынка видеоигр.
    \newline
    • Конкуренты: обзор основных конкурентов и их продуктов.
    \newline
    • Стиль и арты: визуальное оформление игры, стиль графики и иллюстрации.
    \newline
    • Бизнес-модель: стратегия монетизации игры и планы по её распространению.
    \newline
    Feature-лист- это документ, который акцентирует внимание на уникальных особенностях игры, таких как:
    \newline
    • Детали, которые будут выделены: например, высококачественная графика, инновационные игровые механики или уникальный сюжет.
    \newline
    • Уникальные аспекты: элементы, которые отличают игру от конкурентов и делают её узнаваемой.
    \newline
Эти документы в совокупности образуют общий дизайн-документ, или «диз-док», который служит основой для всех последующих этапов разработки. На начальном этапе команда выбирает подходящий игровой движок из существующих на рынке или принимает решение о разработке собственного. В качестве источника вдохновения и ориентира собираются референсы — изображения из реальной жизни, других игр или фильмов, которые затем предоставляются продюсерам и инвесторам для лучшего понимания концепции. На основе этих материалов концепт-художники приступают к созданию визуальных иллюстраций, которые помогут визуализировать идеи команды и представить их в наиболее привлекательном свете.


\subsection{Прототипирование}

Прототипирование в игровой индустрии — это искусство и наука одновременно. Это первый шаг в реализации творческой концепции, который позволяет разработчикам экспериментировать с идеями и механиками, не вкладывая значительные ресурсы в полноценное производство. Прототипирование игры — это как создание эскиза будущего шедевра, где каждая линия и каждый цвет имеют значение, но ещё не окончательно определены.
Прототипирование начинается с формирования базовой концепции игры. Разработчики собираются вместе, чтобы обсудить идеи и выбрать те, которые наиболее перспективны для дальнейшей разработки. Затем, используя различные инструменты и технологии, такие как Unity, Unreal Engine или даже простые инструменты, такие как карточные игры или бумажные прототипы, команда создаёт первоначальную версию игры.
Этот прототип не обязательно должен быть красивым или полностью функциональным. Главное — это возможность проверить основные игровые механики и увидеть, как они взаимодействуют друг с другом. Прототип может включать в себя базовые элементы геймплея, примитивные графические элементы и простейшие звуковые эффекты. Цель здесь — быстро и недорого проверить, насколько весело и интересно будет играть в будущую игру.
Прототипирование играет ключевую роль в процессе разработки, поскольку оно позволяет:\newline
    • Избежать ошибок: Оно помогает выявить потенциальные проблемы на раннем этапе, когда их ещё легко исправить.\newline
    • Сэкономить время и ресурсы: Быстрое тестирование идеи без необходимости создания полноценной игры.\newline
    • Получить обратную связь: Прототип можно представить игрокам, чтобы собрать отзывы и понять, что нужно улучшить.\newline
    • Способствовать креативности: Прототипирование даёт свободу для экспериментов и тестирования самых смелых идей.\newline
    • Оценка визуального оформления и стиля игры. \newline
    • Проверка концепции и идей игры \newline
Прототипирование — это не просто создание модели игры, это процесс итераций, тестирования и улучшений, который продолжается до тех пор, пока разработчики не будут уверены, что их концепция готова к следующему этапу. Это фундаментальный этап, который определяет, будет ли игра интересной, увлекательной и способной привлечь внимание аудитории. В конечном счёте, прототипирование — это о том, как превратить мечту в реальность, шаг за шагом приближаясь к созданию уникального игрового опыта.



\subsection{Создание персонажей и уровней}

При разработке компьютерной игры, первым шагом является определение типа персонажа – может это быть животное, монстр или человек. Затем художники создают мудборды, объединяющие различные референсы, после чего приступают к созданию концепт-эскиза. Далее, модель передается 3D-моделлеру для начала процесса скульптинга.\newline

Скульптинг – это процесс, похожий на лепку из пластилина, который обычно выполняется в программе ZBrush. На этом этапе создается общая форма персонажа, детализируется анатомия и силуэт. Затем следуют этапы заполнения внутренних деталей, работы над крупными и мелкими деталями анатомии, и, в некоторых случаях, сверхдетализация с добавлением пор и морщин. После этого идет ретопология модели и создание UV-карты для текстурирования.\newline

После завершения текстурирования модель переносится в другой 3D-пакет, часто используется 3DS Max, где создается скелет, применяется анимация. Сложные движения заимствуются из реальных действий с помощью специальных датчиков и переносятся на модель. Затем персонаж интегрируется в игровой движок, где проводится настройка анимаций, взаимодействия с искусственным интеллектом и добавление диалоговых возможностей.\newline

Программисты пишут базовый игровой код. Их работа начинается с описания. Команда обговаривает то, что нужно сделать. Затем составляется блок-схема. В ней конкретные части алгоритма изображаются в виде разных геометрических фигур, прямоугольников, параллелограммов, ромбов. Блоки соединяются между собой линиями, которые указывают на последовательность выполнения программы.\newline

После компоновки блоков кода программисты приступают к практической работе. Обычно применяют языки программирования C# и C++  Для графики подключаются библиотеки Direct X и Vulkan, Open GL.
В играх широко используются методы объектно-ориентированного программирования (ООП). Как это работает? Например, при создании оружия, нужно определить базовый класс с его характеристиками, такими как урон, дальность стрельбы и количество патронов, а затем наследовать и дополнять его для различных видов оружия.
Для упрощения разработки, внутри движка используется визуальное программирование с блок-схемами, называемыми «узлами« (нодами). Для мобильных игр используются Java и Swift, а для браузерных – HTML5, PHP и JavaScript.\newline
Геймплей-программисты воплощают в жизнь все игровые механики. Это все то, что оживляет мир и позволяет игроку взаимодействовать с ним. AI-программист создает искусственный интеллект – модель поведения ботов. Здесь также часто используют ноды вместо того, чтобы писать код. Это значительно ускоряет процесс разработки.

Нодовая система удобна тем, что с ней могут спокойно работать и дизайнеры, которые могут не знать программирования. Есть еще сетевые программисты, программисты анимации, интерфейсов и т. д.
По итогу работы всех специалистов создается играбельная демоверсия, которая тестируется на предмет геймплея и оптимизации.

\newline
\subsection{Процесс создания звуковых эффектов и музыки в компьютерных играх}
Звуковой дизайн в компьютерных играх является сложным и многоуровневым процессом, который требует не только технических навыков, но и творческого подхода. Он включает в себя созданиесинхронных шумов,звуковых эффектови музыки, каждый из которых играет свою роль в формировании уникального звукового ландшафта игры.
\newline
1. Синхронные шумы (foley):Синхронные шумы — это звуки, которые создаются для имитации реальных действий в игре. Они могут быть записаны в студии или в полевых условиях. Саунд-дизайнеры часто прибегают к использованию различных предметов, чтобы воссоздать нужные звуки. Например, для имитации звуков шагов по снегу могут использоваться мука или крахмал, а при создании звуков убийства и  противников в играх, они используют фрукты, овощи и орехи, чтобы получить  запоминающиеся звуки. . Эти звуки должны быть убедительными и соответствовать визуальному контенту, чтобы игроки могли полностью погрузиться в игровой мир.
\newline
2. Звуковые эффекты (SFX):Звуковые эффекты — это элементы, которые добавляют динамику и напряженность игровым сценам. Они могут быть как реалистичными, так и фантастическими, в зависимости от жанра игры. Создание SFX — это процесс, который может включать синтезирование звуков или их обработку с помощью специального программного обеспечения. Эти эффекты помогают подчеркнуть ключевые моменты в игре, такие как взрывы, магические заклинания или звуки оружия.
\newline
3. Музыка:Музыкальное сопровождение в играх играет важную роль в создании атмосферы и передаче эмоций. Композиторы работают над созданием музыкальных тем, которые могут быть как постоянными, так и изменяться в зависимости от действий игрока и сюжетных поворотов. Адаптивная музыка реагирует на игровые события, изменяясь таким образом, чтобы отражать текущее состояние игры и поддерживать настроение.
В целом, звуковой дизайн в компьютерных играх — это искусство создания звуковой среды, которая усиливает визуальные и игровые аспекты, делая игру более погруженной и эмоционально насыщенной. Это требует от саунд-дизайнеров не только технической экспертизы, но и креативности, чтобы создать уникальный и запоминающийся звуковой опыт.
\newline


\subsection{Графическое оформление и тестирование}

Этап графического оформления игры является одним из ключевых этапов в процессе разработки, поскольку качественная графика способна значительно повысить привлекательность и атмосферу игры. В процессе графического оформления обычно участвуют художники, дизайнеры и аниматоры.

Как правило, начинают с создания концепт-артов, которые помогают определить стиль и внешний вид игры. Затем художники разрабатывают текстуры для объектов, персонажей, фонов и других элементов игрового мира. Аниматоры работают над анимациями персонажей и объектов, чтобы придать им жизнь.

На этапе графического оформления также разрабатывается пользовательский интерфейс (UI, User Interface), который включает в себя меню, кнопки, индикаторы и другие элементы, обеспечивающие удобство игрового процесса.

Параллельно с этим процессом идет тестирование игры. Тестирование является важным шагом перед выпуском игры, поскольку позволяет выявить ошибки, недочеты и проблемы с производительностью, а также обеспечить качество игрового процесса.

На этапе тестирования проводятся различные виды тестов:
1. Техническое тестирование: проверка игры на различных платформах и устройствах, а также тестирование производительности.
\newline
2. Функциональное тестирование: проверка работоспособности игровых механик, интерфейса, управления и других игровых элементов.
\newline
3. Тестирование на баги: выявление и исправление ошибок, глюков и недочетов, которые могут влиять на игровой процесс.
\newline
4. Игровое тестирование: оценка геймплея, баланса игры, сложности уровней и других аспектов игрового процесса.
\newline
После успешного завершения этапа тестирования игра готовится к релизу, который может включать в себя дополнительные шаги, такие как маркетинговая кампания и продвижение игры.
\section{Тенденции и перспективы развития индустрии компьютерных игр}
\subsection{Влияние новых технологий: виртуальная реальность, AI и др.}

Индустрия компьютерных игр является одной из быстроразвивающихся и наиболее инновационных сфер в современном мире. С появлением новых технологий, таких как виртуальная реальность (VR), искусственный интеллект (ИИ), расширенная реальность (AR) и блокчейн, игровая индустрия находится в процессе значительного изменения и преобразования. Ниже приведены основные тенденции и перспективы развития индустрии компьютерных игр:
\newline
    1. Виртуальная реальность (VR): VR технология предоставляет игрокам возможность погрузиться в игровой мир и взаимодействовать с ним на более глубоком уровне. Разработчики игр все более активно внедряют элементы виртуальной реальности в свои проекты, что делает игровой процесс более захватывающим и реалистичным.
\newline
    2. Искусственный интеллект (ИИ): Технология искусственного интеллекта используется для улучшения геймплея, создания реалистичных противников, автоматизации процессов разработки игр и персонализации игрового опыта для каждого игрока. Благодаря ИИ, игры становятся более адаптивными и уникальными для каждого пользователя.
\newline
    3. Облако и стриминг: С развитием облачных технологий и стриминга игр, игроки могут получить доступ к играм на любом устройстве без необходимости скачивания и установки игровых файлов. Это открывает новые возможности для игровой индустрии, такие как игры на мобильных устройствах с графикой, сравнимой с консолями.
\newline
    4. Киберспорт и онлайн-игры: Киберспорт приобретает все большую популярность и становится профессиональным спортом. Это открывает новые возможности для развития игровой индустрии, включая специализированные турниры, онлайн-трансляции и сотрудничество с крупными спонсорами.
\newline
    5. Блокчейн и цифровые активы: Технология блокчейн используется для создания уникальных цифровых активов в играх, таких как виртуальные предметы, персонажи и скины. Это открывает новые возможности для создания экономических моделей внутри игровых миров и создания перспективных игровых систем.

Общий вывод – индустрия компьютерных игр продолжает активно развиваться благодаря внедрению новых технологий, улучшению игрового опыта и расширению аудитории. Влияние технологий, таких как виртуальная реальность, искусственный интеллект и облачных сервисов, открывает новые горизонты для развития игровой индустрии и создания более увлекательных и инновационных игр.

Перспективы развития профессии разработчика компьютерных игр

Разработчики игр востребованы в различных отраслях, не ограничиваясь только индустрией видеоигр и развлечений. Их спрос есть и в сфере информационных технологий, масс-медиа и других областях, где требуется создание игрового контента. Для того чтобы начать работу гейм-девелопером, желательно пройти специализированные курсы, поскольку самостоятельное обучение может быть менее эффективным из-за сложности профессии. Хороший разработчик компьютерных игр должен владеть несколькими языками программирования, чтобы создавать качественные и инновационные проекты. Профессия разработчика игр достаточно сложная и требует творческого подхода, поэтому она наиболее подходит тем, кто увлечен миром игр и готов уделять им время и усилия. Однако, нулевому уровню также можно обучиться при наличии достаточного усердия и следования качественной образовательной программе. В целом, перспективы развития профессии разработчика компьютерных игр остаются высокими и востребованными в современном мире технологий.

\subsection{Востребованность разработчиков игр}

Спрос на специалистов по разработке компьютерных игр растет по ряду причин. С развитием игровой индустрии увеличивается не только количество игроков, но и продажи игр, что способствует стремительному росту отрасли. Многообразие доступных платформ — от ПК и консолей до мобильных устройств и VR-технологий — требует наличия специалистов, способных создавать игры под каждую из них. Кроме того, игры стали средством выражения, искусства и образования, и обсуждаются на различных уровнях - от новостей до политики.

Обучение через игры с использованием геймификации применяется в образовательных симуляторах для мотивации пользователей. Сама разработка игр становится все сложнее, так как внедрение новых технологий, таких как AI и VR, требует наличие готовых к работе специалистов. 

Экономический потенциал игровых проектов огромен, они приносят миллиардные доходы, привлекая инвестиции и повышая спрос на специалистов. Кроме того, принципы создания игр применяются в маркетинге, образовании и здравоохранении для увеличения мотивации пользователей. Учитывая все эти факторы, становится очевидно, что профессия разработчика игр востребована сегодня и будет пользоваться ростом в будущем.
% После введения — серии \section, \subsection и т.д.

\conclusion
Мысли о будущем развитии компьютерных игр.\newline

Будущего компьютерных игр открывает захватывающие перспективы, где виртуальная (VR) и дополненная реальность (AR) становятся ключевыми элементами, преображая игровые пространства в еще более захватывающие и погружающие миры. Эти технологии обещают не просто новый уровень взаимодействия, но и возможность переживать игровые события, как будто вы находитесь непосредственно внутри игры, ощущая каждый элемент виртуального мира.\newline
Искусственный интеллект (AI) и машинное обучение (ML) уже сейчас начинают революционизировать игровую индустрию, и мы можем ожидать, что их роль станет только усиливаться. Они будут способствовать созданию NPC, которые не просто выполняют заранее заданные функции, но и способны адаптироваться и реагировать на действия игроков непредсказуемо и убедительно, делая игровой мир живым и динамичным.
Кроссплатформенность — это еще один тренд, который будет продолжать развиваться, позволяя игрокам переносить свой игровой опыт между различными устройствами. Это означает, что ваши достижения и прогресс в игре будут с вами, независимо от того, играете ли вы на консоли, ПК или мобильном устройстве, обеспечивая непрерывность и удобство игрового процесса.
Наконец, социальные и экологические аспекты начинают играть все более значимую роль в создании игрового контента. Игры становятся не просто средством развлечения, но и платформой для осмысления и обсуждения актуальных мировых проблем. Они могут повышать осведомленность об экологических вызовах, социальной справедливости и других важных вопросах, стимулируя игроков к размышлениям и действиям.\newline

Обобщение основных моментов.

Мир разработки компьютерных игр является началом путешествия по многоуровневому ландшафту геймдизайна, где каждый жанр — это отдельная вселенная, полная возможностей и открытий. От первых шагов в истории развития игр, мы узнаем о том, как простые текстовые приключения и аркады с минимальной графикой превратились в сложные многопользовательские проекты, предлагающие глубокие и захватывающие истории.
Обзор различных жанров раскрывает, как каждый из них предлагает свои уникальные вызовы и возможности для разработчиков: от стратегий и квестов до симуляторов и экшенов. Это многообразие жанров отражает широкий спектр человеческих эмоций и интересов, предоставляя игрокам возможность выбирать именно те миры и истории, которые им по душе.
Технологии создания игр — это фундамент, на котором строится каждая игра. Проектирование концепции игры — это не просто выбор темы и сюжета, это тщательная работа по созданию уникального игрового мира, который будет живым и интересным для игрока. Программирование — это сердце игры, где код становится заклинанием, оживляющим все элементы игры. Графическое оформление не менее важно, ведь именно оно создает первое впечатление и помогает игроку погрузиться в атмосферу игры. Наконец, тестирование игры — это этап, на котором выявляются все недочеты и ошибки, чтобы в итоге игрок получил максимально положительный опыт.

Разработка игр — это сложный и многогранный процесс, требующий совместных усилий множества специалистов. Каждый этап разработки игры имеет решающее значение для создания конечного продукта, который будет радовать игроков своим качеством и увлекательностью. Именно благодаря этому процессу мир видеоигр постоянно развивается, предлагая всё новые и новые формы развлечений, которые удивляют и вдохновляют миллионы людей по всему миру.
% Отобразить все источники. Даже те, на которые нет ссылок.
% \nocite{*}
\cite{is1}
\cite{is2}
\cite{is3}
\cite{is4}
\cite{is5}
\cite{is6}
\cite{is7}
\cite{is8}
\cite{is9}
\cite{is10}
\cite{is11}

\conclusion

% Отобразить все источники. Даже те, на которые нет ссылок.
% \nocite{*}

\bibliographystyle{ugost2003}
\bibliography{thesis}

% Окончание основного документа и начало приложений Каждая последующая секция
% документа будет являться приложением
\appendix

\end{document}
