\section{Этапы разработки компьютерных игр}
\subsection{Концепция и документация}

На заре разработки новой игры, команда разработчиков, возглавляемая опытным гейм-дизайнером, собирается вместе, чтобы провести серию творческих сессий. Цель этих сессий - сформировать и закрепить фундаментальную концепцию игры, которая будет служить основой для всех последующих этапов разработки. Во время этих встреч обсуждаются и определяются ключевые элементы проекта, такие как жанр игры, её сценарий, игровая механика, персонажи, а также множество других критически важных деталей, которые в конечном итоге определят уникальность и привлекательность игры для целевой аудитории.
Для того чтобы зафиксировать все принятые решения и идеи, разработчики используют три основных типа документации: концепт-документ, Vision-документ и Feature-лист. Эти документы выполняют различные функции и служат разным целям в процессе разработки.
Концепт-документ- это своего рода краткое содержание игры, которое включает в себя следующие пункты:
    • Название игры: краткое и запоминающееся название, которое отражает суть игры.\newline
    • Ключевая концепция: основная идея, лежащая в основе игры, её «душа».\newline
    • Жанр: определение жанра, к которому будет принадлежать игра, что помогает определить стилистические и механические особенности.\newline
    • Целевая аудитория: демографические и психографические характеристики потенциальных игроков.\newline
    • Уникальные особенности: то, что отличает игру от конкурентов и привлекает внимание игроков.\newline
    • Сюжет: общее описание сюжетной линии и ключевых поворотов событий.\newline
    • Механика игры: описание основных игровых механик и правил.\newline
    • Описание игрового мира и персонажей: детальное описание мира, в котором происходят события игры, и характеристики персонажей.\newline
    • Платформы: перечень платформ, на которых будет доступна игра.\newline
    • Системные требования: технические характеристики, необходимые для комфортной игры.\newline
    • График и бюджет разработки: план работ и расчет финансовых затрат на проект.\newline
    • Технологии: обзор технологий и инструментов, которые будут использоваться в процессе создания игры.\newline

    Vision-документ- это более глубокое погружение в мир создаваемой игры, который не столько касается самого игрового процесса, сколько того, какой окончательный продукт хочет получить команда. В этом документе содержится:\newline
    • Полное описание игры: расширенное описание всех аспектов игры, включая механику, сюжет, мир и персонажей.
    \newline
    • Целевая аудитория и исследование рынка: анализ потенциальных игроков и текущего состояния рынка видеоигр.
    \newline
    • Конкуренты: обзор основных конкурентов и их продуктов.
    \newline
    • Стиль и арты: визуальное оформление игры, стиль графики и иллюстрации.
    \newline
    • Бизнес-модель: стратегия монетизации игры и планы по её распространению.
    \newline
    Feature-лист- это документ, который акцентирует внимание на уникальных особенностях игры, таких как:
    \newline
    • Детали, которые будут выделены: например, высококачественная графика, инновационные игровые механики или уникальный сюжет.
    \newline
    • Уникальные аспекты: элементы, которые отличают игру от конкурентов и делают её узнаваемой.
    \newline
Эти документы в совокупности образуют общий дизайн-документ, или «диз-док», который служит основой для всех последующих этапов разработки. На начальном этапе команда выбирает подходящий игровой движок из существующих на рынке или принимает решение о разработке собственного. В качестве источника вдохновения и ориентира собираются референсы — изображения из реальной жизни, других игр или фильмов, которые затем предоставляются продюсерам и инвесторам для лучшего понимания концепции. На основе этих материалов концепт-художники приступают к созданию визуальных иллюстраций, которые помогут визуализировать идеи команды и представить их в наиболее привлекательном свете.


\subsection{Прототипирование}

Прототипирование в игровой индустрии — это искусство и наука одновременно. Это первый шаг в реализации творческой концепции, который позволяет разработчикам экспериментировать с идеями и механиками, не вкладывая значительные ресурсы в полноценное производство. Прототипирование игры — это как создание эскиза будущего шедевра, где каждая линия и каждый цвет имеют значение, но ещё не окончательно определены.
Прототипирование начинается с формирования базовой концепции игры. Разработчики собираются вместе, чтобы обсудить идеи и выбрать те, которые наиболее перспективны для дальнейшей разработки. Затем, используя различные инструменты и технологии, такие как Unity, Unreal Engine или даже простые инструменты, такие как карточные игры или бумажные прототипы, команда создаёт первоначальную версию игры.
Этот прототип не обязательно должен быть красивым или полностью функциональным. Главное — это возможность проверить основные игровые механики и увидеть, как они взаимодействуют друг с другом. Прототип может включать в себя базовые элементы геймплея, примитивные графические элементы и простейшие звуковые эффекты. Цель здесь — быстро и недорого проверить, насколько весело и интересно будет играть в будущую игру.
Прототипирование играет ключевую роль в процессе разработки, поскольку оно позволяет:\newline
    • Избежать ошибок: Оно помогает выявить потенциальные проблемы на раннем этапе, когда их ещё легко исправить.\newline
    • Сэкономить время и ресурсы: Быстрое тестирование идеи без необходимости создания полноценной игры.\newline
    • Получить обратную связь: Прототип можно представить игрокам, чтобы собрать отзывы и понять, что нужно улучшить.\newline
    • Способствовать креативности: Прототипирование даёт свободу для экспериментов и тестирования самых смелых идей.\newline
    • Оценка визуального оформления и стиля игры. \newline
    • Проверка концепции и идей игры \newline
Прототипирование — это не просто создание модели игры, это процесс итераций, тестирования и улучшений, который продолжается до тех пор, пока разработчики не будут уверены, что их концепция готова к следующему этапу. Это фундаментальный этап, который определяет, будет ли игра интересной, увлекательной и способной привлечь внимание аудитории. В конечном счёте, прототипирование — это о том, как превратить мечту в реальность, шаг за шагом приближаясь к созданию уникального игрового опыта.



\subsection{Создание персонажей и уровней}

При разработке компьютерной игры, первым шагом является определение типа персонажа – может это быть животное, монстр или человек. Затем художники создают мудборды, объединяющие различные референсы, после чего приступают к созданию концепт-эскиза. Далее, модель передается 3D-моделлеру для начала процесса скульптинга.\newline

Скульптинг – это процесс, похожий на лепку из пластилина, который обычно выполняется в программе ZBrush. На этом этапе создается общая форма персонажа, детализируется анатомия и силуэт. Затем следуют этапы заполнения внутренних деталей, работы над крупными и мелкими деталями анатомии, и, в некоторых случаях, сверхдетализация с добавлением пор и морщин. После этого идет ретопология модели и создание UV-карты для текстурирования.\newline

После завершения текстурирования модель переносится в другой 3D-пакет, часто используется 3DS Max, где создается скелет, применяется анимация. Сложные движения заимствуются из реальных действий с помощью специальных датчиков и переносятся на модель. Затем персонаж интегрируется в игровой движок, где проводится настройка анимаций, взаимодействия с искусственным интеллектом и добавление диалоговых возможностей.\newline

Программисты пишут базовый игровой код. Их работа начинается с описания. Команда обговаривает то, что нужно сделать. Затем составляется блок-схема. В ней конкретные части алгоритма изображаются в виде разных геометрических фигур, прямоугольников, параллелограммов, ромбов. Блоки соединяются между собой линиями, которые указывают на последовательность выполнения программы.\newline

После компоновки блоков кода программисты приступают к практической работе. Обычно применяют языки программирования C# и C++  Для графики подключаются библиотеки Direct X и Vulkan, Open GL.
В играх широко используются методы объектно-ориентированного программирования (ООП). Как это работает? Например, при создании оружия, нужно определить базовый класс с его характеристиками, такими как урон, дальность стрельбы и количество патронов, а затем наследовать и дополнять его для различных видов оружия.
Для упрощения разработки, внутри движка используется визуальное программирование с блок-схемами, называемыми «узлами« (нодами). Для мобильных игр используются Java и Swift, а для браузерных – HTML5, PHP и JavaScript.\newline
Геймплей-программисты воплощают в жизнь все игровые механики. Это все то, что оживляет мир и позволяет игроку взаимодействовать с ним. AI-программист создает искусственный интеллект – модель поведения ботов. Здесь также часто используют ноды вместо того, чтобы писать код. Это значительно ускоряет процесс разработки.

Нодовая система удобна тем, что с ней могут спокойно работать и дизайнеры, которые могут не знать программирования. Есть еще сетевые программисты, программисты анимации, интерфейсов и т. д.
По итогу работы всех специалистов создается играбельная демоверсия, которая тестируется на предмет геймплея и оптимизации.

\newline
\subsection{Процесс создания звуковых эффектов и музыки в компьютерных играх}
Звуковой дизайн в компьютерных играх является сложным и многоуровневым процессом, который требует не только технических навыков, но и творческого подхода. Он включает в себя созданиесинхронных шумов,звуковых эффектови музыки, каждый из которых играет свою роль в формировании уникального звукового ландшафта игры.
\newline
1. Синхронные шумы (foley):Синхронные шумы — это звуки, которые создаются для имитации реальных действий в игре. Они могут быть записаны в студии или в полевых условиях. Саунд-дизайнеры часто прибегают к использованию различных предметов, чтобы воссоздать нужные звуки. Например, для имитации звуков шагов по снегу могут использоваться мука или крахмал, а при создании звуков убийства и  противников в играх, они используют фрукты, овощи и орехи, чтобы получить  запоминающиеся звуки. . Эти звуки должны быть убедительными и соответствовать визуальному контенту, чтобы игроки могли полностью погрузиться в игровой мир.
\newline
2. Звуковые эффекты (SFX):Звуковые эффекты — это элементы, которые добавляют динамику и напряженность игровым сценам. Они могут быть как реалистичными, так и фантастическими, в зависимости от жанра игры. Создание SFX — это процесс, который может включать синтезирование звуков или их обработку с помощью специального программного обеспечения. Эти эффекты помогают подчеркнуть ключевые моменты в игре, такие как взрывы, магические заклинания или звуки оружия.
\newline
3. Музыка:Музыкальное сопровождение в играх играет важную роль в создании атмосферы и передаче эмоций. Композиторы работают над созданием музыкальных тем, которые могут быть как постоянными, так и изменяться в зависимости от действий игрока и сюжетных поворотов. Адаптивная музыка реагирует на игровые события, изменяясь таким образом, чтобы отражать текущее состояние игры и поддерживать настроение.
В целом, звуковой дизайн в компьютерных играх — это искусство создания звуковой среды, которая усиливает визуальные и игровые аспекты, делая игру более погруженной и эмоционально насыщенной. Это требует от саунд-дизайнеров не только технической экспертизы, но и креативности, чтобы создать уникальный и запоминающийся звуковой опыт.
\newline


\subsection{Графическое оформление и тестирование}

Этап графического оформления игры является одним из ключевых этапов в процессе разработки, поскольку качественная графика способна значительно повысить привлекательность и атмосферу игры. В процессе графического оформления обычно участвуют художники, дизайнеры и аниматоры.

Как правило, начинают с создания концепт-артов, которые помогают определить стиль и внешний вид игры. Затем художники разрабатывают текстуры для объектов, персонажей, фонов и других элементов игрового мира. Аниматоры работают над анимациями персонажей и объектов, чтобы придать им жизнь.

На этапе графического оформления также разрабатывается пользовательский интерфейс (UI, User Interface), который включает в себя меню, кнопки, индикаторы и другие элементы, обеспечивающие удобство игрового процесса.

Параллельно с этим процессом идет тестирование игры. Тестирование является важным шагом перед выпуском игры, поскольку позволяет выявить ошибки, недочеты и проблемы с производительностью, а также обеспечить качество игрового процесса.

На этапе тестирования проводятся различные виды тестов:
1. Техническое тестирование: проверка игры на различных платформах и устройствах, а также тестирование производительности.
\newline
2. Функциональное тестирование: проверка работоспособности игровых механик, интерфейса, управления и других игровых элементов.
\newline
3. Тестирование на баги: выявление и исправление ошибок, глюков и недочетов, которые могут влиять на игровой процесс.
\newline
4. Игровое тестирование: оценка геймплея, баланса игры, сложности уровней и других аспектов игрового процесса.
\newline
После успешного завершения этапа тестирования игра готовится к релизу, который может включать в себя дополнительные шаги, такие как маркетинговая кампания и продвижение игры.